%%%%%%%%%%%%%%%%%%%%%%%%%%%%%%%%%%%Präambel%%%%%%%%%%%%%%%%%%%%%%%%%%%%%%%%%%%%

%%%%%%%%%%%%%%%%%%%%% 
%      Pakete       %
%%%%%%%%%%%%%%%%%%%%%
\usepackage[utf8]{inputenc} % Textcodierung UTF-8
\usepackage[english]{babel} % Deutsche Lokalisierung
\usepackage[T1]{fontenc} % Zeichensatzkodierung
\usepackage[scaled]{helvet} % Helvetica Schriftart
\usepackage{nicefrac} % Darstellung von Brüchen in Formeln
\usepackage{alphalph} % Option mit Buchstaben anstatt mit Zahlen zu zählen (z.B. für Seitenzahlen)
\usepackage{array}
\usepackage{amssymb}
\usepackage{amsmath} % Darstellung von Formeln
\usepackage{amsthm} % Darstellung von Mathematischen Theoremen
\usepackage{threeparttable} % Einbindung von Tabellen mit Fußnoten
\usepackage{caption} % Customise Bildunter- und Tabellenüberschriften
\usepackage{tocloft} % Formatierung von Inhaltsverzeichnis, Abbildungs- und Tabellenverzeichnis
\usepackage{booktabs} % Customise Tabellen mit Extra-Kommandos
\usepackage{enumitem} % Formatierung von Listen (itemize, enumerate and description)
\usepackage{geometry} % Einstellung der Seitengeometrie
\usepackage{siunitx} % SI-Einheiten Paket
\usepackage{color} % Für Matlab Code
\usepackage{colortbl} % Benutzung von Farben in Tabellen
\usepackage{setspace} % Zeilenabstand
\usepackage{multirow} % Tabellen Paket
\usepackage{chngcntr} % Nummerierungen
\usepackage{natbib} % Bibliographie Paket
\usepackage{graphicx} % Zur Einbindung von Abbildungen
\usepackage{epstopdf} % Möglichkeit auch .eps Dateien einzubinden
\usepackage[hyphens]{url} % URL-Paket mit Option zur Zeilentrennung nach Gedanken-/Bindestrichen
\usepackage{hyperref} % Zur Erzeugung von hyperlinks im Dokument
\usepackage[raggedleft]{titlesec} % Formatierung von Kapitel-Überschriften
\usepackage[toc,page]{appendix} % Customise den Anhang
\usepackage{listings} % Einbinden von Listen
\usepackage{nomencl} % Nomenklatur 
\usepackage{ifthen} % Für Bedingungen
\usepackage{fancyhdr} % Header & Footer Package
\usepackage[boxed,chapter]{algorithm} % Einbindung von Algorithmen als Pseudo-Code (schwarze Box um den Pseudo-Code und Nummerierung mit Einbeziehung der Kapitel Nummer
\usepackage{algpseudocode}% http://ctan.org/pkg/algorithmicx
\usepackage{tabto} % Tabulatoren
\usepackage{parskip} % Kontrolle der Absatzgröße
\usepackage[absolute]{textpos} % Positionierung

%%%%%%%%%%%%%%%%%%%%% 
%   Einstellungen   %
%%%%%%%%%%%%%%%%%%%%%
\geometry{a4paper, top=30mm, bottom=35mm, inner=35mm, outer=25mm, headsep=11mm, footskip=21mm} % Seitenformatierung

\renewcommand*\familydefault{\sfdefault} % Schriftart-Einstellung: Sans serif
\DeclareOldFontCommand{\bf}{\normalfont\bfseries}{\mathbf} % define command \bf  (used by bibliography I think)
\linespread{1.5}\selectfont % Zeilenabstand
\sloppy % Silbentrennungeinstellung: möglichst selten trennen

%Einstellungen für Seitenumbrüche  
\interfootnotelinepenalty=9999 % Kein Seitenumbruch innerhalb einer Fußnote!
\clubpenalty0
\widowpenalty0
\displaywidowpenalty0

% Kapitel und Unterkapitel Formatierung
\titleformat{\chapter}{\Large\bfseries}{\Large\bfseries\thechapter}{6pt}{\Large}
\titleformat{\section}{\large\bfseries}{\large\bfseries\thesection}{6pt}{\large}
\titleformat{\subsection}{\normalsize\bfseries}{\normalsize\bfseries\thesubsection}{6pt}{\normalsize}
\titlespacing*{\chapter}{0pt}{-35pt}{0pt} % Vertikale Abstände
\titlespacing*{name=\chapter,numberless}{0pt}{-35pt}{0pt}
\titlespacing*{\section}{0pt}{0pt}{0pt}
\titlespacing*{\subsection}{0pt}{0pt}{0pt}

% Abbildungs-/Tabelleneinstellungen
\captionsetup{labelfont={bf}} % Abbildung/Tabelle fett drucken

% SI-Einheit Einstellungen
\sisetup{per-mode = symbol}%
\sisetup{detect-weight = true}

% Definition von Befehlen, um Unterkapitel des Anhangs nicht ins Inhaltsverzeichnis aufzunehmen
\newcommand{\stoptocwriting}{\addtocontents{toc}{\protect\setcounter{tocdepth}{-5}}} % Stopt die Aufnahme von Kapitelnamen ins Inhaltsverzeichnis
\newcommand{\resumetocwriting}{\addtocontents{toc}{\protect\setcounter{tocdepth}{\arabic{tocdepth}}}} % Nimmt die Aufnahme von Kapitelnamen ins Inhaltsverzeichnis wieder auf
  
% Einbindung von Matlab-Code mit den Matlab-Farben
\definecolor{mlgreen}{rgb}{.035,.6,.251}
\definecolor{mlviolett}{rgb}{.643,.259,.804}
\lstdefinestyle{mlab}{language=Matlab, 
otherkeywords={},
deletekeywords={def,real,length,gamma,zeros,figure,subplot,plot,xlabel,ylabel,zlabel,sin,pi,diff,size,meshgrid,legend,surf,min,trapz,cos,eig,exp,abc},
numbers=left,
numberstyle=\tiny,%5
basicstyle={\ttfamily},%
keywordstyle={\color{blue}},%
commentstyle=\color{mlgreen},%
stringstyle=\color{mlviolett},%
breaklines=true,%
showstringspaces=false,%
numberbychapter=true, %
xleftmargin=15pt
 }

% Einstellungen für das Symbolverzeichnis
\newlength{\nomgroupstartsep}
\setlength{\nomgroupstartsep}{16pt}

\newlength{\nomintermsep}
\setlength{\nomintermsep}{-\parsep}


\renewcommand{\nomgroup}[1]{%
	\ifthenelse{\equal{#1}{A}}{\item[\textbf{Abbreviations}]}{%
	\ifthenelse{\equal{#1}{I}}{\vspace{16pt} \item[\textbf{Indices}]}{%
	\ifthenelse{\equal{#1}{G}}{\vspace{16pt} \item[\textbf{Greek Symbols}]}{%
	\ifthenelse{\equal{#1}{L}}{\vspace{16pt} \item[\textbf{Latin Symbols}]}}}}

}
\setlength{\nomlabelwidth}{.2\hsize} % Abstand zwischen Variable und Erklärung
\setlength{\nomitemsep}{-\parsep} % Zeilenabstand verkleinern
\renewcommand{\nomlabel}[1]{#1 \dotfill} % Punkte zw. Abkürzung und Erklärung
\newcommand{\nomunit}[1]{\renewcommand{\nomentryend}{\dotfill #1}}
\makenomenclature % Erstellen des Symbolverzeichnisses

% Einstellung Hyperlink
\hypersetup{colorlinks,citecolor=black,filecolor=black,linkcolor=black,urlcolor=black} % Hyperlink Einstellungen

% Einstellung Algorithmus Package
\floatname{algorithm}{Algorithmus} % Unterschrift eines Algorithmus ist auf Deutsch

% Header & Footer Einstellungen
  \pagestyle{fancy} %eigener Seitenstil
\renewcommand{\chaptermark}[1]{\markboth{\thechapter~ #1}{}}
\renewcommand{\sectionmark}[1]{\markright{\thechapter~ #1}}

  \fancyhf{}%alle Kopf- und Fußzeilenfelder bereinigen
  \renewcommand{\headrulewidth}{0.4pt}%obere Trennlinie
  \fancyhead[R]{\small{\nouppercase\leftmark}}
  \renewcommand{\footrulewidth}{0.4pt}%untere Trennlinie
  \fancyfoot[L]{\small{Institute of Aircraft Design $\quad |\quad$  Technical University of Munich}}
  \fancyfoot[R]{\thepage} %Seitennummer

	\fancypagestyle{plain}{
  \fancyhf{}
  \renewcommand{\headrulewidth}{0.4pt}
  \fancyhead[R]{\small{\nouppercase\leftmark}}
  \renewcommand{\footrulewidth}{0.4pt}
  \fancyfoot[L]{\small{Institute of Aircraft Design $\quad |\quad$  Technical University of Munich}}
  \fancyfoot[R]{\thepage}
}

\edef\restoreparindent{\parindent=\the\parindent\relax}
\restoreparindent

